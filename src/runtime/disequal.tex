\nwfilename{disequal.nw}\nwbegindocs{0}% -*- noweb-code-mode: c-mode -*-% ===> this file was generated automatically by noweave --- better not edit it
% $Id: disequal.nw,v 2.15 2004/05/02 09:17:27 wlux Exp $
%
% Copyright (c) 2002-2004, Wolfgang Lux
% See ../LICENSE for the full license.
%
\subsection{Disequality Constraints}
Disequality constraints extend the common constraint structure by the
term from which the variable has to be different.

\nwenddocs{}\nwbegincode{1}\sublabel{NWg1BHq-4B9WEF-1}\nwmargintag{{\nwtagstyle{}\subpageref{NWg1BHq-4B9WEF-1}}}\moddef{disequal.h~{\nwtagstyle{}\subpageref{NWg1BHq-4B9WEF-1}}}\endmoddef\nwstartdeflinemarkup\nwprevnextdefs{\relax}{NWg1BHq-4B9WEF-2}\nwenddeflinemarkup
extern NodeInfo diseq_info;
typedef struct diseq_constraint \{
    Constraint cstr;
    Node       *node;
\} Disequality;

\nwalsodefined{\\{NWg1BHq-4B9WEF-2}}\nwnotused{disequal.h}\nwendcode{}\nwbegindocs{2}\nwdocspar
The primitive function \texttt{=/=} implements disequality constraints.

\nwenddocs{}\nwbegincode{3}\sublabel{NWg1BHq-4B9WEF-2}\nwmargintag{{\nwtagstyle{}\subpageref{NWg1BHq-4B9WEF-2}}}\moddef{disequal.h~{\nwtagstyle{}\subpageref{NWg1BHq-4B9WEF-1}}}\plusendmoddef\nwstartdeflinemarkup\nwprevnextdefs{NWg1BHq-4B9WEF-1}{\relax}\nwenddeflinemarkup
DECLARE_ENTRYPOINT(___61__47__61_);

\nwendcode{}\nwbegindocs{4}\nwdocspar
\nwenddocs{}\nwbegincode{5}\sublabel{NWg1BHq-28youZ-1}\nwmargintag{{\nwtagstyle{}\subpageref{NWg1BHq-28youZ-1}}}\moddef{disequal.c~{\nwtagstyle{}\subpageref{NWg1BHq-28youZ-1}}}\endmoddef\nwstartdeflinemarkup\nwprevnextdefs{\relax}{NWg1BHq-28youZ-2}\nwenddeflinemarkup
#include "config.h"
#include "debug.h"
#include <stdio.h>
#include <stdlib.h>
#include <string.h>
#include "run.h"
#include "heap.h"
#include "stack.h"
#include "eval.h"
#include "threads.h"
#include "spaces.h"
#include "unify.h"
#include "disequal.h"
#include "trail.h"
#include "cam.h"
#include "trace.h"

#define pair_cons_node_size constr_node_size(3)
static
NodeInfo pair_cons_info = \{
    CONS_TAG, pair_cons_node_size, (const int *)0, (Label)eval_whnf, ",:", (FinalFun)0
\};

DECLARE_LABEL(___61__47__61__1);
DECLARE_LABEL(___61__47__61__2);
DECLARE_LABEL(diseq_data);
DECLARE_LABEL(diseq_var);
DECLARE_LABEL(diseq_args);
#if YIELD_NONDET
DECLARE_LABEL(diseq_args_1);
#endif
DECLARE_LABEL(diseq_args_2);
DECLARE_LABEL(diseq_args_3);
DECLARE_LABEL(diseq_args_4);
DECLARE_LABEL(diseq_args_5);
DECLARE_LABEL(check_diseq);

#define diseq_constraint_size wordsof(Disequality)
NodeInfo diseq_constraint_info = \{
    0, diseq_constraint_size, (const int *)0, (Label)check_diseq, (const char *)0,
    (FinalFun)0
\};

FUNCTION(___61__47__61_)
\{
    EXPORT_LABEL(___61__47__61_)
 ENTRY_LABEL(___61__47__61_)

    TRACE(("%I enter =/=%V\\n", 2, sp));
    GOTO(___61__47__61__1);
\}

static
FUNCTION(___61__47__61__1)
\{
    Node *aux;

 ENTRY_LABEL(___61__47__61__1)
    EVAL_FLEX_POLY(___61__47__61__1);
    aux   = sp[0];
    sp[0] = sp[1];
    sp[1] = aux;
    GOTO(___61__47__61__2);
\}

static
FUNCTION(___61__47__61__2)
\{
    unsigned int n;
    double       d, e;
    Node         *arg1, *arg2;

 ENTRY_LABEL(___61__47__61__2)
    EVAL_FLEX_POLY(___61__47__61__2);

    arg1 = sp[1];
    arg2 = sp[0];

    while ( is_boxed(arg1) && is_indir_node(arg1) )
        arg1 = arg1->n.node;
    if ( is_boxed(arg1) && is_variable_node(arg1) )
    \{
        /* check for trivial disequality (x=/=x) */
        if ( arg1 == arg2 )
            FAIL();
        sp[0] = arg1;
        sp[1] = arg2;
        GOTO(diseq_var);
    \}
    else if ( is_boxed(arg2) && is_variable_node(arg2) )
    \{
        sp[0] = arg2;
        sp[1] = arg1;
        GOTO(diseq_var);
    \}
#if !ONLY_BOXED_OBJECTS
    else if ( is_unboxed(arg1) )
    \{
        ASSERT(is_unboxed(arg2));
        if ( arg1 == arg2 )
            FAIL();
    \}
#endif
    else
    \{
        ASSERT(is_boxed(arg2));
        if ( node_tag(arg1) == node_tag(arg2) )
            switch ( node_tag(arg1) )
            \{
            case CHAR_TAG:
                if ( arg1->ch.ch == arg2->ch.ch )
                    FAIL();
                break;
#if ONLY_BOXED_OBJECTS
            case INT_TAG:
                if ( arg1->i.i == arg2->i.i )
                    FAIL();
                break;
#endif
            case FLOAT_TAG:
                get_float_val(d, arg1->f);
                get_float_val(e, arg2->f);
                if ( d == e )
                    FAIL();
                break;

            case PAPP_TAG:
                if ( arg1->info == arg2->info )
                \{
                    if ( closure_argc(arg1) > 0 )
                    \{
                        sp[0] = arg1;
                        sp[1] = arg2;
                        GOTO(diseq_data);
                    \}
                    FAIL();
                \}
                break;

            case SEARCH_CONT_TAG:
                if ( arg1 == arg2 )
                    FAIL();
                break;

            default:
                ASSERT(is_constr_node(arg1) || is_abstract_node(arg1));
                if ( is_abstract_node(arg1) )
                \{
                    if ( arg1 != arg2 )
                        break;
                    n = 0;
                \}
                else if ( is_vector(arg1) )
                \{
                    if ( arg1->a.length != arg2->a.length )
                        break;
                    n = vector_argc(arg1);
                \}
                else
                    n = constr_argc(arg1);
                if ( n > 0 )
                \{
                    sp[0] = arg1;
                    sp[1] = arg2;
                    GOTO(diseq_data);
                \}
                FAIL();
        \}
    \}

    sp += 2;
    RETURN(Success);
\}

\nwalsodefined{\\{NWg1BHq-28youZ-2}\\{NWg1BHq-28youZ-3}\\{NWg1BHq-28youZ-4}\\{NWg1BHq-28youZ-5}}\nwnotused{disequal.c}\nwendcode{}\nwbegindocs{6}\nwdocspar
A disequality constraint between a variable and another expression is
handled by evaluating the expression to normal form and adding the
resulting data term to the constraint list of the variable. In
order to ensure that a constraint is always a data term, the runtime
system should create a new variable and unify this variable with the
argument (which also causes evaluation to normal form). However, we
add the expression directly to constraint list here, which should not
cause any problems as the runtime system does not rely on disequality
constraints being in normal form.

\ToDo{Avoid adding redundant disequalities by checking the constraint
lists before adding a new constraint.}

\nwenddocs{}\nwbegincode{7}\sublabel{NWg1BHq-28youZ-2}\nwmargintag{{\nwtagstyle{}\subpageref{NWg1BHq-28youZ-2}}}\moddef{disequal.c~{\nwtagstyle{}\subpageref{NWg1BHq-28youZ-1}}}\plusendmoddef\nwstartdeflinemarkup\nwprevnextdefs{NWg1BHq-28youZ-1}{NWg1BHq-28youZ-3}\nwenddeflinemarkup
static
FUNCTION(diseq_var)
\{
    Node        *aux;
    Disequality *cstr;

 ENTRY_LABEL(diseq_var)
    if ( !is_local_space(sp[0]->v.spc) )
    \{
        if ( is_boxed(sp[1]) && is_variable_node(sp[1])
             && is_local_space(sp[1]->v.spc) )
        \{
            aux   = sp[0];
            sp[0] = sp[1];
            sp[1] = aux;
        \}
        else
            GOTO(delay_thread(___61__47__61__1, sp[0]));
    \}

    if ( occurs(sp[0], sp[1]) )
    \{
        sp += 2;
        RETURN(Success);
    \}

    /* add the constraint to the variable */
    CHECK_HEAP(diseq_constraint_size);
    cstr             = (Disequality *)hp;
    cstr->cstr.info  = &diseq_constraint_info;
    cstr->cstr.cstrs = sp[0]->v.cstrs;
    cstr->node       = sp[1];
    hp              += diseq_constraint_size;

    SAVE(sp[0], v.cstrs);
    sp[0]->v.cstrs = (Constraint *)cstr;

    /* evaluate the argument to normal form */
    sp[0] = sp[1];
    GOTO(___61__58__61_);
\}

\nwendcode{}\nwbegindocs{8}\nwdocspar
The function {\Tt{}diseq{\_}data\nwendquote} handles disequalities between two data terms
with arguments and the same constructor at the root. Such a disequality
can be solved only by introducing a disjunction into the program. The
disequality
\begin{quote}
$c\;e_1\;\dots\;e_n$ \texttt{=/=} $c\;e_1'\;\dots\;e_n'$
\end{quote}
is equivalent of the disjunction
\begin{quote}
$e_1$\texttt{=/=}$e_1'$ \texttt{|}                                           %'
$e_2$\texttt{=/=}$e_2'$ \texttt{|} \dots \texttt{|}                          %'
$e_n$\texttt{=/=}$e_n'$                                                      %'
\end{quote}
where \texttt{|} denotes the disjunction operator. However, we replace
the disequality by the more complex disjunction
\begin{quote}
$e_1$\texttt{=/=}$e_1'$ \texttt{|}                                           %'
($e_1$\texttt{=:=}$e_1'$ \texttt{\&} $e_2$\texttt{=/=}$e_2'$)
\texttt{|} \dots \texttt{|}
($e_1$\texttt{=:=}$e_1'$ \texttt{\&} $e_2$\texttt{=:=}$e_2'$
 \texttt{\&} \dots \texttt{\&} $e_n$\texttt{=/=}$e_n'$)                      %'
\end{quote}
which avoids generating duplicate solutions.

As a little optimization, {\Tt{}diseq{\_}data\nwendquote} already checks for trivial
disequalities and succeeds immediately if any is found. In addition,
disjuncts where a trivial equality is found are discarded immediately.

We consider a disequality trivial here if both corresponding arguments
are ground atoms. In addition, if both arguments are data constructors
applications with different roots, the disequality will succeed
immediately.

For all remaining argument pairs, {\Tt{}diseq{\_}data\nwendquote} creates a list that
is then processed by {\Tt{}diseq{\_}args\nwendquote} below. If the list is empty, the
disequality constraint cannot be satisfied and {\Tt{}diseq{\_}data\nwendquote} fails
immediately.

\nwenddocs{}\nwbegincode{9}\sublabel{NWg1BHq-28youZ-3}\nwmargintag{{\nwtagstyle{}\subpageref{NWg1BHq-28youZ-3}}}\moddef{disequal.c~{\nwtagstyle{}\subpageref{NWg1BHq-28youZ-1}}}\plusendmoddef\nwstartdeflinemarkup\nwprevnextdefs{NWg1BHq-28youZ-2}{NWg1BHq-28youZ-4}\nwenddeflinemarkup
static
FUNCTION(diseq_data)
\{
    boolean      is_vect;
    unsigned int i, n;
    double       d, e;
    Node         *next, *arglist, *x, *y, **argp1, **argp2;
    word         *oldHp;

 ENTRY_LABEL(diseq_data)

    is_vect = is_vector(sp[0]);
    n       = is_vect ? vector_argc(sp[0]) : constr_argc(sp[0]);

    CHECK_HEAP(n * pair_cons_node_size);
    oldHp = hp;
    argp1 = is_vect ? sp[0]->a.args : sp[0]->c.args;
    argp2 = is_vect ? sp[1]->a.args : sp[1]->c.args;

    arglist = nil;
    argp1  += n;
    argp2  += n;
    for ( i = n; i-- > 0; )
    \{
        x = *--argp1;
        y = *--argp2;
        while ( is_boxed(x) && is_indir_node(x) )
            x = x->n.node;
        while ( is_boxed(y) && is_indir_node(y) )
            y = y->n.node;

#if !ONLY_BOXED_OBJECTS
        if ( is_unboxed(x) )
        \{
            if ( is_unboxed(y) )
            \{
                if ( x != y )
                \{
                    hp  = oldHp;
                    sp += 2;
                    RETURN(Success);
                \}
                continue;
            \}
        \}
        else
#endif /* !ONLY_BOXED_OBJECTS */
            switch ( node_tag(x) )
            \{
            case CHAR_TAG:
                if ( is_char_node(y) )
                \{
                    if ( x->ch.ch != y->ch.ch )
                    \{
                        hp  = oldHp;
                        sp += 2;
                        RETURN(Success);
                    \}
                    continue;
                \}
                break;
#if ONLY_BOXED_OBJECTS
            case INT_TAG:
                if ( is_int_node(y) )
                \{
                    if ( int_val(x) != int_val(y) )
                    \{
                        hp  = oldHp;
                        sp += 2;
                        RETURN(Success);
                    \}
                    continue;
                \}
                break;
#endif /* ONLY_BOXED_OBJECTS */
            case FLOAT_TAG:
                if ( is_float_node(y) )
                \{
                    get_float_val(d, x->f);
                    get_float_val(e, y->f);
                    if ( d != e )
                    \{
                        hp  = oldHp;
                        sp += 2;
                        RETURN(Success);
                    \}
                    continue;
                \}
                break;
            case VARIABLE_TAG:
                if ( x == y )
                    continue;
                break;
            case PAPP_TAG:
                if ( is_papp_node(y) )
                \{
                    if ( x->info != y->info )
                    \{
                        hp  = oldHp;
                        sp += 2;
                        RETURN(Success);
                    \}
                    if ( closure_argc(x) == 0 )
                        continue;
                \}
                break;
            case SEARCH_CONT_TAG:
                if ( is_search_cont_node(y) )
                \{
                    if ( x != y )
                    \{
                        hp  = oldHp;
                        sp += 2;
                        RETURN(Success);
                    \}
                    continue;
                \}
            case CLOSURE_TAG:
            case SUSPEND_TAG:
            case QUEUEME_TAG:
                break;
            default:
                ASSERT(is_constr_node(x) || is_abstract_node(x));
                if ( is_constr_node(y) || is_abstract_node(y) )
                \{
                    if ( node_tag(x) != node_tag(y) )
                    \{
                        hp  = oldHp;
                        sp += 2;
                        RETURN(Success);
                    \}
                    if ( is_abstract_node(x) )
                    \{
                        if ( x != y )
                        \{
                            hp  = oldHp;
                            sp += 2;
                            RETURN(Success);
                        \}
                        continue;
                    \}
                    else if ( is_vector(x) )
                    \{
                        if ( x->a.length != y->a.length )
                        \{
                            hp  = oldHp;
                            sp += 2;
                            RETURN(Success);
                        \}
                        if ( vector_argc(x) == 0 )
                            continue;
                    \}
                    else if ( constr_argc(x) == 0 )
                        continue;
                    break;
                \}
                break;
            \}

        next            = (Node *)hp;
        next->c.info    = &pair_cons_info;
        next->c.args[0] = x;
        next->c.args[1] = y;
        next->c.args[2] = arglist;
        arglist         = next;
        hp             += pair_cons_node_size;
    \}

    if ( arglist == nil )
        FAIL();

    *++sp = arglist;
    GOTO(diseq_args);
\}

\nwendcode{}\nwbegindocs{10}\nwdocspar
The function {\Tt{}diseq{\_}args\nwendquote} implements the disequality constraint for
a non-empty list of argument pairs. If the list contains more than one
pair, it creates a disjunction for $x_1$\texttt{=/=}$x_1'$ and               %'
$x_1$\texttt{=:=}$x_1'$ \texttt{\&} \texttt{diseq\_args} $xs$ $xs'$.

\ToDo{Find a better implementation that splits a disequality
constraint for $n$ argument pairs immediately into $n$ disjuncts.}

\nwenddocs{}\nwbegincode{11}\sublabel{NWg1BHq-28youZ-4}\nwmargintag{{\nwtagstyle{}\subpageref{NWg1BHq-28youZ-4}}}\moddef{disequal.c~{\nwtagstyle{}\subpageref{NWg1BHq-28youZ-1}}}\plusendmoddef\nwstartdeflinemarkup\nwprevnextdefs{NWg1BHq-28youZ-3}{NWg1BHq-28youZ-5}\nwenddeflinemarkup
static Label diseq_args_choices[] = \{ diseq_args_2, diseq_args_3, (Label)0 \};

static
FUNCTION(diseq_args)
\{
 ENTRY_LABEL(diseq_args)
    ASSERT(sp[0]->info == &pair_cons_info);

    if ( sp[0]->c.args[2] == nil )
        GOTO(diseq_args_2);

#if YIELD_NONDET
    if ( rq != (ThreadQueue)0 )
        GOTO(yield_thread(diseq_args_1));
#endif
    choice_conts = diseq_args_choices;
    GOTO(nondet_handlers.choices);
\}

#if YIELD_NONDET
static
FUNCTION(diseq_args_1)
\{
 ENTRY_LABEL(diseq_args_1)
    choice_conts = diseq_args_choices;
    GOTO(nondet_handlers.choices);
\}
#endif

static
FUNCTION(diseq_args_2)
\{
    Node *arglist;

 ENTRY_LABEL(diseq_args_2)
    CHECK_STACK1();
    arglist = sp[0];
    sp     -= 1;
    sp[0]   = arglist->c.args[0];
    sp[1]   = arglist->c.args[1];
    GOTO(___61__47__61_);
\}

static
FUNCTION(diseq_args_3)
\{
    Node *susp, *arglist;

 ENTRY_LABEL(diseq_args_3)
    CHECK_STACK(6);
    CHECK_HEAP(queueMe_node_size);

    arglist = sp[0];

    susp        = (Node *)hp;
    susp->info  = &queueMe_info;
    susp->q.wq  = (ThreadQueue)0;
    susp->q.spc = ss;
    hp         += queueMe_node_size;

    sp   -= 6;
    sp[0] = arglist->c.args[0];
    sp[1] = arglist->c.args[1];
    sp[2] = (Node *)update;
    sp[3] = susp;
    sp[4] = (Node *)diseq_args_4;
    sp[5] = susp;
    sp[6] = arglist->c.args[2];
    start_thread(5);
    GOTO(___61__58__61_);
\}

static
FUNCTION(diseq_args_4)
\{
    Node *r;

 ENTRY_LABEL(diseq_args_4)
    for ( r = sp[0]; node_tag(r) == INDIR_TAG; r = r->n.node )
        ;

    if ( node_tag(r) == SUCCESS_TAG )
        sp++;
    else
    \{
        ASSERT(node_tag(r) == QUEUEME_TAG);
        CHECK_STACK1();
        sp   -= 1;
        sp[0] = sp[2];
        sp[1] = (Node *)diseq_args_5;
        sp[2] = r;
    \}
    GOTO(diseq_args);
\}

static
FUNCTION(diseq_args_5)
\{
    Node *r;

 ENTRY_LABEL(diseq_args_5)
    ASSERT(node_tag(sp[0]) == SUCCESS_TAG);

    for ( r = sp[1]; node_tag(r) == INDIR_TAG; r = r->n.node )
        ;
    if ( node_tag(r) == QUEUEME_TAG )
    \{
        *++sp = r;
        GOTO(r->info->eval);
    \}
    ASSERT(node_tag(r) == SUCCESS_TAG);

    sp += 2;
    RETURN(r);
\}

\nwendcode{}\nwbegindocs{12}\nwdocspar
The function {\Tt{}check{\_}diseq\nwendquote} is used to check that a value is compatible
with a given constraint.

\nwenddocs{}\nwbegincode{13}\sublabel{NWg1BHq-28youZ-5}\nwmargintag{{\nwtagstyle{}\subpageref{NWg1BHq-28youZ-5}}}\moddef{disequal.c~{\nwtagstyle{}\subpageref{NWg1BHq-28youZ-1}}}\plusendmoddef\nwstartdeflinemarkup\nwprevnextdefs{NWg1BHq-28youZ-4}{\relax}\nwenddeflinemarkup
static
FUNCTION(check_diseq)
\{
 ENTRY_LABEL(check_diseq)
    sp[1] = ((Disequality *)sp[1])->node;
    GOTO(___61__47__61_);
\}
\nwendcode{}

\nwixlogsorted{c}{{disequal.c}{NWg1BHq-28youZ-1}{\nwixd{NWg1BHq-28youZ-1}\nwixd{NWg1BHq-28youZ-2}\nwixd{NWg1BHq-28youZ-3}\nwixd{NWg1BHq-28youZ-4}\nwixd{NWg1BHq-28youZ-5}}}%
\nwixlogsorted{c}{{disequal.h}{NWg1BHq-4B9WEF-1}{\nwixd{NWg1BHq-4B9WEF-1}\nwixd{NWg1BHq-4B9WEF-2}}}%

