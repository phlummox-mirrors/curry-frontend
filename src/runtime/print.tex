\nwfilename{print.nw}\nwbegindocs{0}% -*- noweb-code-mode: c-mode -*-% ===> this file was generated automatically by noweave --- better not edit it
% $Id: print.nw,v 2.9 2004/02/13 19:25:06 wlux Exp $
%
% Copyright (c) 2001-2003, Wolfgang Lux
% See ../LICENSE for the full license.
%
\subsection{Displaying the result}
When the computation finishes, the computed result and the bindings of
all global variables can be printed with the function {\Tt{}print{\_}result\nwendquote}.

\nwenddocs{}\nwbegincode{1}\sublabel{NW62yNM-2O23p1-1}\nwmargintag{{\nwtagstyle{}\subpageref{NW62yNM-2O23p1-1}}}\moddef{print.h~{\nwtagstyle{}\subpageref{NW62yNM-2O23p1-1}}}\endmoddef\nwstartdeflinemarkup\nwenddeflinemarkup
extern void print_result(const char *var_names[], Node *vars[], Node *result);

\nwnotused{print.h}\nwendcode{}\nwbegindocs{2}\nwdocspar
\nwenddocs{}\nwbegincode{3}\sublabel{NW62yNM-LrMVL-1}\nwmargintag{{\nwtagstyle{}\subpageref{NW62yNM-LrMVL-1}}}\moddef{print.c~{\nwtagstyle{}\subpageref{NW62yNM-LrMVL-1}}}\endmoddef\nwstartdeflinemarkup\nwprevnextdefs{\relax}{NW62yNM-LrMVL-2}\nwenddeflinemarkup
#include "config.h"
#include <stdio.h>
#include <stdlib.h>
#include <string.h>
#include "debug.h"
#include "run.h"
#include "heap.h"
#include "trail.h"
#include "disequal.h"
#include "vars.h"
#include "char.h"
#include "print.h"

static void    print_node(unsigned, Node *);
static void    print_app(unsigned, Node *, unsigned, Node **);
static boolean is_string(Node *);
static void    print_string(Node *);
static void    print_list(Node *);
static void    print_tuple(Node *);
static boolean print_constraints(boolean, Node *);
static boolean print_app_constraints(boolean, unsigned, Node **);
static boolean print_constrained_var(boolean, Node *);

\nwalsodefined{\\{NW62yNM-LrMVL-2}\\{NW62yNM-LrMVL-3}\\{NW62yNM-LrMVL-4}\\{NW62yNM-LrMVL-5}}\nwnotused{print.c}\nwendcode{}\nwbegindocs{4}\nwdocspar
The global function {\Tt{}print{\_}result\nwendquote} will print the computed result
and the bindings of the global variables. Only those variables which
have been bound by the program are actually printed. In order to name
the logical variables, a dictionary is used (see below) which is
initialized before printing the nodes and reset afterwards.

Note that it is safe to enter the names of the free variables of the
goal into the dictionary because these entries are never going to
be released. This is prevented by keeping a pointer to the initial
closure at the bottom of the stack. This closure is used to access the
bindings of the free variables when the computation succeeds.

\nwenddocs{}\nwbegincode{5}\sublabel{NW62yNM-LrMVL-2}\nwmargintag{{\nwtagstyle{}\subpageref{NW62yNM-LrMVL-2}}}\moddef{print.c~{\nwtagstyle{}\subpageref{NW62yNM-LrMVL-1}}}\plusendmoddef\nwstartdeflinemarkup\nwprevnextdefs{NW62yNM-LrMVL-1}{NW62yNM-LrMVL-3}\nwenddeflinemarkup
void
print_result(const char **var_names, Node **vars, Node *result)
\{
    unsigned         i, n;
    boolean          hasAnswer;
    struct dict_node *tail;

    tail      = names_tail;
    hasAnswer = false;
    for ( n = 0; var_names[n] != (const char *)0; n++ )
    \{
        if ( !is_variable_node(vars[n]) )
        \{
            if ( !hasAnswer )
            \{
                printf("\{%s = ", var_names[n]);
                hasAnswer = true;
            \}
            else
                printf(", %s = ", var_names[n]);
            print_node(0, vars[n]);
        \}
    \}

    hasAnswer = print_constraints(hasAnswer, result);
    for ( i = 0; i < n; i++ )
        hasAnswer = print_constraints(hasAnswer, vars[i]);

    if ( hasAnswer )
        printf("\} ");

    print_node(0, result);
    release_names(tail);
\}

\nwendcode{}\nwbegindocs{6}\nwdocspar
When a node is printed, we will in general use the constructor name
found in the info vector of node. Some special kinds of nodes are
handled differently. In particular, for each unbound variable a name
is generated and this name is shown by the printer. Also two special
functions are used for printing lists and tuples. The {\Tt{}prec\nwendquote}
parameter is used to determine when a constructor term has to be
enclosed in parentheses. At present, we distinguish three different
levels; {\Tt{}0\nwendquote} is used for terms at the top-level and arguments of a
tuple or list, {\Tt{}1\nwendquote} is used for arguments within an infix
application, and {\Tt{}2\nwendquote} is used for arguments of a regular application.

\nwenddocs{}\nwbegincode{7}\sublabel{NW62yNM-LrMVL-3}\nwmargintag{{\nwtagstyle{}\subpageref{NW62yNM-LrMVL-3}}}\moddef{print.c~{\nwtagstyle{}\subpageref{NW62yNM-LrMVL-1}}}\plusendmoddef\nwstartdeflinemarkup\nwprevnextdefs{NW62yNM-LrMVL-2}{NW62yNM-LrMVL-4}\nwenddeflinemarkup
static void
print_node(unsigned prec, Node *node)
\{
    char         buf[32], *cp;
    unsigned int i;
    double       d;

    for (;;)
    \{
#if !ONLY_BOXED_OBJECTS
        if ( is_unboxed(node) )
        \{
            printf(prec > 0 && unboxed_val(node) < 0 ? "(%ld)" : "%ld",
                   unboxed_val(node));
            return;
        \}
        else
#endif
            switch ( node_tag(node) )
            \{
            case CHAR_TAG:
                printf("'%s'", lit_char(node->ch.ch, '\\''));
                return;

#if ONLY_BOXED_OBJECTS
            case INT_TAG:
                printf(prec > 0 && node->i.i < 0 ? "(%ld)" : "%ld",
                       node->i.i);
                return;
#endif
            case FLOAT_TAG:
                get_float_val(d, node->f);
                sprintf(buf, "%g", d);

                cp = strpbrk(buf, ".e");
                if ( cp == (char *)0 )
                    strcat(buf, ".0");
                else if ( *cp != '.'  )
                \{
                    i = strlen(cp) + 1;
                    for ( cp += i; i-- > 0; cp-- )
                        cp[2] = cp[0];
                    cp[1] = '.';
                    cp[2] = '0';
                \}
                printf(prec > 0 && buf[0] == '-' ? "(%s)" : "%s", buf);
                return;
            case PAPP_TAG:
            case CLOSURE_TAG:
                print_app(prec, node, closure_argc(node), node->cl.args);
                return;
            case VARIABLE_TAG:
                printf("%s", lookup_name(node));
                return;
            case SUSPEND_TAG:
                node = node->s.fn;
                break;
            case QUEUEME_TAG:
                printf("Suspended");
                return;
            case INDIR_TAG:
                node = node->n.node;
                break;
            default:
                if ( is_abstract_node(node) || is_search_cont_node(node) )
                \{
                    const char *name = node->info->cname;
                    if ( name == (const char *)0 )
                        name = "<abstract>";
                    printf("%s", name);
                \}
                else if ( node->info == (NodeInfo *)&cons_info )
                \{
                    if ( is_string(node) )
                        print_string(node);
                    else
                        print_list(node);
                \}
                else if ( is_tuple(node->info) )
                    print_tuple(node);
                else if ( is_vector(node) )
                    print_app(prec, node, vector_argc(node), node->a.args);
                else
                    print_app(prec, node, constr_argc(node), node->c.args);
                return;
            \}
    \}
\}

static void
print_app(unsigned prec, Node *node, unsigned argc, Node **argv)
\{
    unsigned i;
    boolean  isop   = is_operator(node->info);
    boolean  infix  = isop && argc == 2;
    boolean  parens = infix ? prec > 0 : prec > 1 && argc != 0;

    if ( parens )
        putchar('(');

    if ( infix )
    \{
        print_node(1, argv[0]);
        printf(" %s ", node->info->cname);
        print_node(1, argv[1]);
    \}
    else
    \{
        printf(isop ? "(%s)" : "%s", node->info->cname);

        for ( i = 0; i < argc ; i++ )
        \{
            putchar(' ');
            print_node(2, argv[i]);
        \}
    \}

    if ( parens )
        putchar(')');
\}

\nwendcode{}\nwbegindocs{8}\nwdocspar
The list printer shows lists in the usual list notation of Curry. If
the tail of the list is not an empty list it will be printed in
Prolog style, i.e., the tail is separated from the rest of the list by
a vertical bar instead of a comma.

If the list is ground and consists only of characters, we use the
string notation for display. Note that we may need to traverse the
whole list for this before we can start printing. Fortunately, we
know that the printing code is applied only to finite data terms.

\nwenddocs{}\nwbegincode{9}\sublabel{NW62yNM-LrMVL-4}\nwmargintag{{\nwtagstyle{}\subpageref{NW62yNM-LrMVL-4}}}\moddef{print.c~{\nwtagstyle{}\subpageref{NW62yNM-LrMVL-1}}}\plusendmoddef\nwstartdeflinemarkup\nwprevnextdefs{NW62yNM-LrMVL-3}{NW62yNM-LrMVL-5}\nwenddeflinemarkup
static boolean
is_string(Node *list)
\{
    Node *head;

    while ( list->info == &cons_info )
    \{
        head = list->c.args[0];
        while ( is_boxed(head) && head->info->tag == INDIR_TAG )
            head = head->n.node;
        if ( is_unboxed(head) || head->info->tag != CHAR_TAG )
            return false;

        list = list->c.args[1];
        while ( list->info->tag == INDIR_TAG )
            list = list->n.node;
    \}
    
    return list == nil;
\}

static void
print_string(Node *list)
\{
    Node *head;

    putchar('"');
    while ( list->info == &cons_info )
    \{
        head = list->c.args[0];
        while ( is_boxed(head) && head->info->tag == INDIR_TAG )
            head = head->n.node;
        ASSERT(is_boxed(head) && head->info->tag == CHAR_TAG);
        printf("%s", lit_char(head->ch.ch, '"'));

        list = list->c.args[1];
        while ( list->info->tag == INDIR_TAG )
            list = list->n.node;
   \}
   ASSERT(list == nil);
   putchar('"');
\}

static void
print_list(Node *list)
\{
    char sep = '[';

    while ( list->info == &cons_info )
    \{
        putchar(sep);
        print_node(0, list->c.args[0]);
        sep = ',';

        list = list->c.args[1];
        while ( list->info->tag == INDIR_TAG )
            list = list->n.node;
    \}

    if ( list != nil )
    \{
        putchar('|');
        print_node(0, list);
    \}

    putchar(']');
\}

static void
print_tuple(Node *node)
\{
    unsigned int i, n;
    boolean      first = true;
    Node         **argv;

    putchar('(');

    if ( is_vector(node) )
    \{
        n    = vector_argc(node);
        argv = node->a.args;
    \}
    else
    \{
        n    = constr_argc(node);
        argv = node->c.args;
    \}
    for ( i = 0; i < n; i++ )
    \{
        if ( first )
            first = false;
        else
            printf(",");
        print_node(0, argv[i]);
    \}

    putchar(')');
\}

\nwendcode{}\nwbegindocs{10}\nwdocspar
The printer traverses the result and all free variables and outputs
all constraints it encounters. In order to print the constraints for a
variable only once, the constraint field of the variable node is reset
before the constraint is printed. The constraint itself is saved on
the trail so that it is restored when the runtime system backtracks to
compute the next solution.

\nwenddocs{}\nwbegincode{11}\sublabel{NW62yNM-LrMVL-5}\nwmargintag{{\nwtagstyle{}\subpageref{NW62yNM-LrMVL-5}}}\moddef{print.c~{\nwtagstyle{}\subpageref{NW62yNM-LrMVL-1}}}\plusendmoddef\nwstartdeflinemarkup\nwprevnextdefs{NW62yNM-LrMVL-4}{\relax}\nwenddeflinemarkup
static boolean
print_constraints(boolean hasAnswer, Node *node)
\{
    for (;;)
    \{
        if ( is_boxed(node) )
            switch ( node_tag(node) )
            \{
            case PAPP_TAG:
            case CLOSURE_TAG:
                hasAnswer = print_app_constraints(hasAnswer,
                                                  closure_argc(node),
                                                  node->cl.args);
                break;
            case VARIABLE_TAG:
                if ( node->v.cstrs != (Constraint *)0 )
                    hasAnswer = print_constrained_var(hasAnswer, node);
                break;
            case SUSPEND_TAG:
                node = node->s.fn;
                continue;
            case INDIR_TAG:
                node = node->n.node;
                continue;
            default:
                if ( is_constr_node(node) )
                \{
                    if ( is_vector(node) )
                        hasAnswer = print_app_constraints(hasAnswer,
                                                          vector_argc(node),
                                                          node->a.args);
                    else
                        hasAnswer = print_app_constraints(hasAnswer,
                                                          constr_argc(node),
                                                          node->c.args);
                \}
                break;
            \}
        break;
    \}
    return hasAnswer;
\}

static boolean
print_app_constraints(boolean hasAnswer, unsigned argc, Node **argv)
\{
    unsigned i;

    for ( i = 0; i < argc; i++ )
        hasAnswer = print_constraints(hasAnswer, argv[i]);

    return hasAnswer;
\}

static boolean
print_constrained_var(boolean hasAnswer, Node *var)
\{
    Constraint *cstrs;

    ASSERT(is_variable_node(var) && var->v.cstrs != (Constraint *)0);

    cstrs = var->v.cstrs;
    SAVE(var, v.cstrs);
    var->v.cstrs = (Constraint *)0;

    for ( ; cstrs != (Constraint *)0; cstrs = cstrs->cstrs )
    \{
        if ( !hasAnswer )
        \{
            printf("\{");
            hasAnswer = true;
        \}
        else
            printf(", ");
        print_node(0, var);
        printf(" /= ");
        print_node(0, ((Disequality *)cstrs)->node);

        hasAnswer = print_constraints(hasAnswer, ((Disequality *)cstrs)->node);
    \}

    return hasAnswer;
\}
\nwendcode{}

\nwixlogsorted{c}{{print.c}{NW62yNM-LrMVL-1}{\nwixd{NW62yNM-LrMVL-1}\nwixd{NW62yNM-LrMVL-2}\nwixd{NW62yNM-LrMVL-3}\nwixd{NW62yNM-LrMVL-4}\nwixd{NW62yNM-LrMVL-5}}}%
\nwixlogsorted{c}{{print.h}{NW62yNM-2O23p1-1}{\nwixd{NW62yNM-2O23p1-1}}}%

