\nwfilename{backtrack.nw}\nwbegindocs{0}% -*- noweb-code-mode: c-mode -*-% ===> this file was generated automatically by noweave --- better not edit it
% $Id: backtrack.nw,v 2.8 2004/05/03 09:25:24 wlux Exp $
%
% Copyright (c) 2002-2004, Wolfgang Lux
% See ../LICENSE for the full license.
%
\subsection{Global search}
In the global search space, the abstract machine uses backtracking for
the implementation of non-deterministic evaluations. This is due to
the fact the a simple backtracking implementation is more efficient
than a depth-first search strategy implemented on top of the
encapsulated search.

There is only one entry-point to this file, viz. the function
{\Tt{}curry{\_}eval\nwendquote} which evaluates its arguments and prints all solutions
to the standard output. This function initializes the abstract machine
and then evaluates all solutions of the goal using backtracking.

The names of all free variables to which the goal is applied before
the evaluation are passed as second argument to {\Tt{}curry{\_}eval\nwendquote} in a
(0-terminated) array.

\nwenddocs{}\nwbegincode{1}\sublabel{NWgnyzV-1aSaXD-1}\nwmargintag{{\nwtagstyle{}\subpageref{NWgnyzV-1aSaXD-1}}}\moddef{backtrack.h~{\nwtagstyle{}\subpageref{NWgnyzV-1aSaXD-1}}}\endmoddef\nwstartdeflinemarkup\nwenddeflinemarkup
extern int curry_eval(FunctionInfo *goal_info_table, const char *fv_names[],
                      int argc, char **argv);

\nwnotused{backtrack.h}\nwendcode{}\nwbegindocs{2}\nwdocspar
\nwenddocs{}\nwbegincode{3}\sublabel{NWgnyzV-3ecTlL-1}\nwmargintag{{\nwtagstyle{}\subpageref{NWgnyzV-3ecTlL-1}}}\moddef{backtrack.c~{\nwtagstyle{}\subpageref{NWgnyzV-3ecTlL-1}}}\endmoddef\nwstartdeflinemarkup\nwprevnextdefs{\relax}{NWgnyzV-3ecTlL-2}\nwenddeflinemarkup
#include "curry.h"
#include <unistd.h>
#include <ctype.h>
#include "print.h"
#include "vars.h"
#include "stats.h"

static void push_choicepoint(Label *);

DECLARE_LABEL(start);
DECLARE_LABEL(stop);
DECLARE_LABEL(choices);
DECLARE_LABEL(fail);

\nwalsodefined{\\{NWgnyzV-3ecTlL-2}\\{NWgnyzV-3ecTlL-3}}\nwnotused{backtrack.c}\nwendcode{}\nwbegindocs{4}\nwdocspar
In this file we also provide the handler functions for
non-deterministic instructions, suspensions and failures. The handler
invoked by the \texttt{Choices} instruction allocates a choicepoint on
the control stack. The saved continuation will execute the next
alternative either with an updated choicepoint or after dropping the
choicepoint, depending on how many alternatives remain.

\nwenddocs{}\nwbegincode{5}\sublabel{NWgnyzV-3ecTlL-2}\nwmargintag{{\nwtagstyle{}\subpageref{NWgnyzV-3ecTlL-2}}}\moddef{backtrack.c~{\nwtagstyle{}\subpageref{NWgnyzV-3ecTlL-1}}}\plusendmoddef\nwstartdeflinemarkup\nwprevnextdefs{NWgnyzV-3ecTlL-1}{NWgnyzV-3ecTlL-3}\nwenddeflinemarkup
static struct nondet_handlers global_handlers = \{ choices, stop, fail \};

static void
push_choicepoint(Label *alts)
\{
    Choicepoint        *oldBp = bp;
    const unsigned int cp_sz  = wordsof(Choicepoint);
    const unsigned int stk_sz =
        (bp != (Choicepoint *)0 ? (Node **)bp : stack_end) - sp;

    /* create a new choicepoint on the stack */
    CHECK_STACK(cp_sz + stk_sz);
    bp           = (Choicepoint *)(sp - cp_sz);
    bp->btAlts   = alts;
    bp->btCid    = cid;
    bp->btDsBase = ds_base;
    bp->btBp     = oldBp;
    bp->btRq     = rq;
    bp->btTp     = tp;
    bp->btDict   = names_tail;
    bp->btHp     = hp;

    /* copy the current thread to the top of the stack */
    memcpy(sp - cp_sz - stk_sz, sp, stk_sz*word_size);
    sp      -= cp_sz + stk_sz;
    ds_base -= cp_sz + stk_sz;

    /* adjust the trail limit */
    hlim = hp;
\}

static
FUNCTION(choices)
\{
 ENTRY_LABEL(choices)

    TRACE(("%I try\\n"));

    /* create a new choicepoint on the control stack */
    push_choicepoint(choice_conts + 1);

    /* continue at the first alternative */
    GOTO(choice_conts[0]);
\}

static
FUNCTION(fail)
\{
    unsigned int stk_sz;
    Label        *choice_conts;

 ENTRY_LABEL(fail)

    TRACE(("%I fail\\n"));

    /* if no alternatives are available terminate the program */
    if ( bp == (Choicepoint *)0 )
    \{
        ASSERT(cid != 0);
        halt();
    \}

    /* restore the old bindings from the trail */
    RESTORE(bp->btTp);

    /* restore registers from the choicepoint */
    cid     = bp->btCid;
    ds_base = bp->btDsBase;
    rq      = bp->btRq;

    /* release the memory allocated since the last choicepoint */
    release_names(bp->btDict);
    release_mem();

    /* if only one alternative remains, drop the choicepoint */
    choice_conts = bp->btAlts;
    ASSERT(choice_conts[0] != (Label)0);
    if ( choice_conts[1] == (Label)0 )
    \{
        TRACE(("%I trust\\n"));
        sp   = (Node **)bp + wordsof(Choicepoint);
        bp   = bp->btBp;
        hlim = bp == (Choicepoint *)0 ? (word *)0 : bp->btHp;
    \}

    /* otherwise update the choicepoint */
    else
    \{
        TRACE(("%I retry\\n"));
        bp->btAlts = choice_conts + 1;

        /* copy the stack */
        stk_sz =
            (bp->btBp != (Choicepoint *)0 ? (Node **)bp->btBp : stack_end)
                - (Node **)(bp + 1);
        sp       = (Node **)bp - stk_sz;
        ds_base -= wordsof(Choicepoint) + stk_sz;
        memcpy(sp, sp + wordsof(Choicepoint) + stk_sz, stk_sz * word_size);
    \}

    /* continue at the next alternative */
    GOTO(choice_conts[0]);
\}

\nwendcode{}\nwbegindocs{6}\nwdocspar
The {\Tt{}curry{\_}eval\nwendquote} function is the main function for a program which
evaluates a (non-monadic) goal. It applies the goal function to a list
of free variables and incrementally computes the non-deterministic
solutions of this application. First, it calls {\Tt{}eval{\_}apply\nwendquote} in order
to create the application node. Next, {\Tt{}eval{\_}first\nwendquote} is used to
initialize the abstract machine for the evaluation of the application
and to compute a first solution. The runtime system assumes that the
goal expression has been transformed into a function
\begin{quote}
  $f$ $x_0$ $x_1$ \dots{} $x_n$ = $x_0$ \texttt{=:=} \emph{goal}
\end{quote}
where $x_1$, \dots, $x_n$ are the free variables of the goal
expression. The unification forces the evaluation of the goal to
normal form. When this function returns successfully, the abstract
machine halts and {\Tt{}curry{\_}eval\nwendquote} prints the result bound to $x_0$
together with the constraints for the free variable $x_1$, \dots,
$x_n$. If there are non-deterministic alternatives to the computed
solution -- i.e., if a choicepoint exists -- and the user chooses to
see more solutions, {\Tt{}eval{\_}next\nwendquote} is used to compute the next solution
by backtracking to the current choicepoint.

When the program is connected to a terminal the user is asked after
every solution if she wants to see more solutions. Otherwise, all
solutions are computed. The interactive and non-interactive operation
can be enforced by passing the options \texttt{-i} and \texttt{-n},
respectively, to the program.

Note that {\Tt{}eval{\_}first\nwendquote} and {\Tt{}eval{\_}next\nwendquote} use {\Tt{}cid\ ==\ 0\nwendquote} in order
to check whether the execution has stopped without a failure. The
{\Tt{}fail\nwendquote} code above does not reset the machine registers; therefore,
if no alternative continuation is available in {\Tt{}fail\nwendquote} and the
machine is halted {\Tt{}cid\nwendquote} contains a valid thread id. On the other
hand, the {\Tt{}stop\nwendquote} code, which is invoked when a deadlock occurs in
global search mode, sets {\Tt{}cid\nwendquote} to {\Tt{}0\nwendquote}. Both {\Tt{}eval{\_}first\nwendquote} and
{\Tt{}eval{\_}next\nwendquote} return the suspended application node that was created
for the goal. It is used to distinguish normal termination from a
deadlock.

\nwenddocs{}\nwbegincode{7}\sublabel{NWgnyzV-3ecTlL-3}\nwmargintag{{\nwtagstyle{}\subpageref{NWgnyzV-3ecTlL-3}}}\moddef{backtrack.c~{\nwtagstyle{}\subpageref{NWgnyzV-3ecTlL-1}}}\plusendmoddef\nwstartdeflinemarkup\nwprevnextdefs{NWgnyzV-3ecTlL-2}{\relax}\nwenddeflinemarkup
static Node    *eval_apply(FunctionInfo *, const char **);
static Node    *eval_first(Node *);
static Node    *eval_next();
static boolean eval_continue(boolean *);
static void    bad_usage(const char *) __attribute__((noreturn));

static void
bad_usage(const char *pname)
\{
    fprintf(stderr, "usage: %s [-i|-n]\\n", pname);
    fprintf(stderr, " -i\\tforce interactive mode\\n");
    fprintf(stderr, " -n\\tforce non-interactive mode\\n");
    exit(1);
\}

int
curry_eval(FunctionInfo *goal_info_table, const char *fv_names[],
           int argc, char **argv)
\{
    int     opt;
    boolean first = true, interactive;
    Node    *susp;
    ADD_LOCAL_ROOTS1((Node *)0);
#define goal LOCAL_ROOT[0]

    /* check if process is connected to a terminal */
    interactive = isatty(0) == 1;

    /* process command line options */
    while ( (opt = getopt(argc, argv, "in")) != EOF )
        switch ( opt )
        \{
        case 'i':
            interactive = true;
            break;
        case 'n':
            interactive = false;
            break;
        default:
            fprintf(stderr, "%s: unknown option -%c\\n", argv[0], opt);
            bad_usage(argv[0]);
        \}

    if ( optind != argc )
    \{
        fprintf(stderr, "%s: too many arguments\\n", argv[0]);
        bad_usage(argv[0]);
    \}

    /* evaluate goal */
    goal = eval_apply(goal_info_table, fv_names);
    susp = eval_first(goal);
    if ( susp != (Node *)0 )
        for ( ;; )
        \{
            if ( !interactive )
            \{
                if ( first )
                    first = false;
                else
                    printf(" | ");
            \}
            if ( is_indir_node(susp) )
                print_result(fv_names, goal->cl.args + 1, goal->cl.args[0]);
            else
                printf("Suspended");

            if ( !eval_continue(&interactive) )
                break;

            susp = eval_next();
            if ( susp == (Node *)0 )
            \{
                printf("%s\\n", interactive ? "No more solutions" : "");
                break;
            \}
        \}
    else
        fprintf(interactive ? stdout : stderr, "No solution\\n");
#undef goal
    DROP_LOCAL_ROOTS();
    return 0;
\}

static Node *
eval_apply(FunctionInfo *goal_info_table, const char *fv_names[])
\{
    unsigned int         i, n;
    const char           **fv;
    Node                 *clos, *var;
    struct variable_node *vars;

    for ( n = 0, fv = fv_names; *fv != (const char *)0; n++, fv++ )
        ;
    ASSERT(goal_info_table[0].arity == n + 1);

    CHECK_HEAP(closure_node_size(n + 1) + (n + 1) * variable_node_size);
    vars = (struct variable_node *)hp;
    for ( i = 0; i <= n; i++ )
    \{
        var          = (Node *)hp;
        var->info    = &variable_info;
        var->v.cstrs = (Constraint *)0;
        var->v.wq    = (ThreadQueue)0;
        var->v.spc   = ss;
        hp          += variable_node_size;
    \}

    clos          = (Node *)hp;
    clos->cl.info = goal_info_table + n + 1;
    for ( i = 0; i <= n; i++ )
        clos->cl.args[i] = (Node *)(vars + i);
    hp += closure_node_size(n + 1);

    for ( i = 0; i < n; i++ )
        add_name((Node *)(vars + i + 1), fv_names[i]);

    return clos;
\}

static Node *
eval_first(Node *goal)
\{
    Node *susp;

    CHECK_HEAP(queueMe_node_size);
    susp        = (Node *)hp;
    susp->info  = &queueMe_info;
    susp->q.wq  = (ThreadQueue)0;
    susp->q.spc = ss;
    hp         += queueMe_node_size;

    sp    = stack_end - 2;
    sp[0] = goal;
    sp[1] = susp;

    run(start);
    return cid == 0 ? stack_end[-1] : (Node *)0;
\}

static Node *
eval_next()
\{
    run(fail);
    return cid == 0 ? stack_end[-1] : (Node *)0;
\}

static boolean
eval_continue(boolean *interactive)
\{
    int c, c1;

    /* terminate the program if no alternatives remain */
    if ( bp == (Choicepoint *)0 )
    \{
        printf("\\n");
        return false;
    \}

    if ( *interactive )
    \{
        printf("\\nMore solutions? [Y(es)/n(o)/a(ll)] ");
        fflush(stdout);
        c = getchar();
        while ( c != EOF && c != '\\n' && isspace(c) )
            c = getchar();
        for ( c1 = c; c1 != EOF && c1 != '\\n'; )
            c1 = getchar();
        if ( c1 == EOF )
            printf("\\n");

        if ( c == 'n' || c == 'N' || c == EOF )
            return false;
        else if ( c == 'a' || c == 'A' )
            *interactive = false;
    \}
    else
        fflush(stdout);

    /* backtrack to the next solution */
    return true;
\}

static
FUNCTION(start)
\{
    Node  *susp, *goal;
    Label eval;
 ENTRY_LABEL(start)
    nondet_handlers = global_handlers;
    TRACE(("start program\\n"));

    goal = sp[0];
    susp = sp[1];
    eval = goal->info->eval;

    CHECK_STACK(3);
    sp   -= 3;
    sp[0] = goal;
    sp[1] = (Node *)update;
    sp[2] = susp;
    sp[3] = (Node *)0;
    start_thread(4);
    GOTO(eval);
\}

static
FUNCTION(stop)
\{
 ENTRY_LABEL(stop)
    cid = 0;
    halt();
\}
\nwendcode{}

\nwixlogsorted{c}{{backtrack.c}{NWgnyzV-3ecTlL-1}{\nwixd{NWgnyzV-3ecTlL-1}\nwixd{NWgnyzV-3ecTlL-2}\nwixd{NWgnyzV-3ecTlL-3}}}%
\nwixlogsorted{c}{{backtrack.h}{NWgnyzV-1aSaXD-1}{\nwixd{NWgnyzV-1aSaXD-1}}}%

